\documentclass[sigconf]{acmart}

\usepackage{balance}	% balancing bibstyles as per request in accepted submission


%% These commands are for a PROCEEDINGS abstract or paper.
\copyrightyear{2021}
\acmYear{2021}
\setcopyright{acmcopyright}
\acmConference[ACM-BCB '21]{ACM-BCB '21: ACM Conference on Bioinformatics, Computational Biology, and Health Informatics}{August 01--04, 2021}{Online}
\acmBooktitle{}
\acmPrice{}
\acmISBN{}
\acmDOI{}

	\author{Vincent Rubinetti}
			\orcid{0000-0002-4655-3773}
				\affiliation{
							\institution{University of Pennsylvania}
										\department{Perelman School of Medicine}
										\city{Philadelphia}
										\state{PA}
										\country{USA}
					}
			\affiliation{
							\institution{University of Colorado School of Medicine}
										\department{Center for Health AI}
										\city{Aurora}
										\state{CO}
										\country{USA}
					}
				\email{vince.rubinetti@gmail.com}
		\author{Anthony Gitter}
			\orcid{0000-0002-5324-9833}
				\affiliation{
							\institution{University of Wisconsin-Madison}
										\department{Department of Biostatistics and Medical Informatics}
														}
			\affiliation{
							\institution{Morgridge Institute for Research}
													\city{Madison}
										\state{WI}
										\country{USA}
					}
				\email{gitter@biostat.wisc.edu}
		\author{COVID-19 Review Consortium}
				\affiliation{
							\institution{None}
																	}
			\author{Casey S. Greene}
			\orcid{0000-0001-8713-9213}
				\affiliation{
							\institution{University of Pennsylvania}
										\department{Department of Systems Pharmacology and Translational Therapeutics}
														}
			\affiliation{
							\institution{Alex's Lemonade Stand Foundation}
										\department{Childhood Cancer Data Lab}
										\city{Philadelphia}
										\state{PA}
										\country{USA}
					}
			\affiliation{
							\institution{University of Colorado School of Medicine}
										\department{Department of Biochemistry and Molecular Genetics}
														}
			\affiliation{
										\department{Center for Health AI}
										\city{Aurora}
										\state{CO}
										\country{USA}
					}
				\email{greenescientist@gmail.com}
		\author{Daniel S. Himmelstein}
			\orcid{0000-0002-3012-7446}
				\affiliation{
							\institution{None}
																	}
				\email{daniel.himmelstein@gmail.com}
		\author{Halie M. Rando}
			\orcid{0000-0001-7688-1770}
				\affiliation{
							\institution{University of Pennsylvania}
										\department{Department of Systems Pharmacology and Translational Therapeutics}
										\city{Philadelphia}
										\state{PA}
										\country{USA}
					}
			\affiliation{
							\institution{University of Colorado School of Medicine}
										\department{Department of Biochemistry and Molecular Genetics}
														}
			\affiliation{
										\department{Center for Health AI}
										\city{Aurora}
										\state{CO}
										\country{USA}
					}
				\email{halie.rando@cuanschutz.edu}
		\author{Lucy D'Agostino McGowan}
			\orcid{0000-0001-7297-9359}
				\affiliation{
							\institution{Wake Forest University}
										\department{Department of Mathematics and Statistics}
										\city{Winston-Salem}
										\state{NC}
										\country{USA}
					}
				\email{lucydagostino@gmail.com}
		\author{Michael P. Robson}
			\orcid{0000-0002-4859-0033}
				\affiliation{
							\institution{Villanova University}
										\department{Department of Computing Sciences}
										\city{Villanova}
										\state{PA}
										\country{USA}
					}
				\email{michael.robson@villanova.edu}
		\author{Ryan Velazquez}
			\orcid{0000-0002-3655-3403}
				\affiliation{
							\institution{Azimuth1}
													\city{McLean}
										\state{VA}
										\country{USA}
					}
				\email{rnhvelazquez@gmail.com}
		\author{Simina M. Boca}
			\orcid{0000-0002-1400-3398}
				\affiliation{
							\institution{Georgetown University Medical Center}
										\department{Innovation Center for Biomedical Informatics}
										\city{Washington}
										\state{DC}
										\country{USA}
					}
				\email{smb310@georgetown.edu}
	

\begin{document}


	\title{An Open-Publishing Response to the COVID-19 Infodemic}




\renewcommand{\shortauthors}{}


\maketitle
\bibliographystyle{ACM-Reference-Format}

	{\let\thefootnote\relax\footnote{Conflicts of interest. Anthony Gitter: Filed a patent application with the Wisconsin Alumni Research Foundation related to classifying activated T cells. Lucy D'Agostino McGowan: Received consulting fees from Acelity and Sanofi in the past five years.}}

\hypertarget{abstract}{%
\section{ABSTRACT}\label{abstract}}

In an effort to keep pace as new information about COVID-19 and SARS-CoV-2 becomes available, this project is an open, collaborative effort that invited contributions from the scientific community broadly, similar to previous efforts to develop collaborative reviews \citep{PZMP42Ak, heDA5StF}.

\hypertarget{ccs-concepts}{%
\section{CCS CONCEPTS}\label{ccs-concepts}}

\hypertarget{keywords}{%
\section{KEYWORDS}\label{keywords}}

\hypertarget{introduction}{%
\section{INTRODUCTION}\label{introduction}}

Coronavirus Disease 2019 (COVID-19) has shaped the years 2020 and 2021 by causing a world-wide public health crisis.
The scientific community has responded by turning significant attention and resources towards COVID-19 and the associated virus, SARS-CoV-2.
The result has been the rapid release of data, results, and publications related to COVID-19 at a scale never previously seen.
Over 20,000 articles about COVID-19 were released in the first 4 months of the pandemic \citep{7ub6VM4Z}.
The velocity and volume of information being released led to the pandemic being termed as an ``infodemic'' as well \citep{7ub6VM4Z, nnfOazAC}.
While this influx of information is likely evidence of important work towards understanding the virus and the disease, there are also downsides to the availability of too much information.
The potential for ``excessive publication'' has been identified as an issue for over forty years, and was one concern about the move towards electronic, rather than print, publishing at the turn of the millennium \citep{DfSr1Ohc}.

Test CORD-19 statistics: 544406 total publications.

While some of this information has been disseminated by traditional publishing mechanisms, in other cases, it is made public through preprint servers or even press releases.
Preprints provide a venue for scientists to release findings rapidly, but have both the advantage and disadvantage of making research available before it has undergone the peer review process.
Media outlets don't always report on this accurately.
Additionally, many papers are being retracted.
These include both preprints and papers that are published in more traditional venues.
The large number of retractions may also be influenced by the fact that the time from submission to peer review for papers related to COVID-19 is very low.

The rate of this proliferation also presents challenges to efforts to summarize and synthesize existing literature, which are necessary given the volume.
A number of groups have sought to track and review COVID-19 preprints.
However, any static review is likely to quickly become dated as new research is released or existing research is retracted or superseded, and the explosive rate of publication made localized efforts to curate new publications increasingly difficult.
Additionally, the complex nature of COVID-19 means that significant advantages can be gained from examining the virus and disease in a multidisciplinary context.
Therefore, the COVID-19 publishing climate presented a challenge where curation of the literature by a diverse group of experts in a format that could respond quickly to high-volume, high-velocity information was desirable.

Recent advances in open publishing have created an infrastructure that facilitates distributed, version-controlled collaboration on manuscripts \citep{YuJbg3zO}.
Manubot \citep{YuJbg3zO} is a collaborative framework developed to adapt open-source software development techniques and version control for manuscript writing.
With Manubot, manuscripts are managed and maintained using GitHub, a popular, online version control interface that also provides the infrastructure via continuous integration (CI) to incorporate code into the manuscript building process to allow, for example, figures to be continuously updated based on an external data set.
This open-publishing platform has been used to develop large-scale collaborative efforts such as .
However, although synthesis and discussion of the emerging literature by biomedical scientists and clinicians would be expected to provide novel insights into how various areas of COVID-19 research intersect, such tools are not typically associated with biomedical research and the reliance on git can present a significant technical barrier to entry for biomedical scientists.
The problem of synthesizing the COVID-19 literature lends itself well to a crowd-sourced approach to writing through open collaboration, but in biology, such efforts often rely on WYSIWYG tools such as Google Docs despite the significant limitations of such approaches.
Therefore, in addition to the unprecedented release of information, COVID-19 presents a unique challenge because most subject matter experts have limited technical training.

Here, we describe efforts to adapt Manubot to handle the extreme case of the COVID-19 infodemic, with the objective of extending simply reviewing preprints to develop a centralized platform for summarizing and synthesizing a massive amount of preprints, news stories, journal publications, and data.
Unlike prior collaborations built on Manubot, here most contributors came from a traditional biological or medical background.
The members of the COVID-19 Review Consortium worked to consolidate information about the virus in the context of related viruses and to synthesize rapidly emerging literature centered on the diagnosis and treatment of COVID-19.
Manubot provided the infrastructure to manage contributions from the community and create a living, scholarly document that integrated data from multiple sources to respond to the COVID-19 crisis in real time and a back-end that allowed biomedical scientists to sort and distill informative content out of the overwhelming flood of information \citep{1HZeeO4Cs} in order to provide a resource that would be useful to the broader scientific community.
This case study demonstrates the value of open collaborative writing tools such as Manubot to emerging challenges and the flexibility of Manubot to be adapted to problems unique to a range of fields.
By recording the evolution of information over time and assembling a resource that auto-updated in response to the evolving crisis, it revealed the particular value that Manubot holds for managing a rapid changes in scientific thought.

\hypertarget{methods}{%
\section{METHODS}\label{methods}}

\hypertarget{contributor-recruitment}{%
\subsection{Contributor Recruitment}\label{contributor-recruitment}}

One of the primary goals of this project was to develop Manubot as a platform accessible to researchers with limited computational training, as is common in biology and medicine.
Given the limitations imposed upon scientists by the COVID-19 pandemic and social distancing measures that had most scientists (including students) working from home for much of 2020, community building across disciplines and across career stages was a priority of the project.
The current project was managed through GitHub \citep{11MwmdOKi} using Manubot \citep{YuJbg3zO} to continuously generate a version of the manuscript online \citep{yTsmmAYC}.
Contributors were recruited by word of mouth and on Twitter, and we sought out opportunities to integrate existing efforts to train early-career researchers (ECRs).
Few researchers in biological and medical fields are trained in version control tools such as git

In order to make the project accessible to individuals from a number of backgrounds, we developed resources explaining how to use GitHub's web interface to develop and edit text and interact with Manubot for individuals with no prior experience working with git or other version control platforms.

Interested parties were encouraged to contribute in a number of ways.
One option was to submit articles of interest as issues in the GitHub repository.
Articles were classified as \emph{diagnostic}, \emph{therapeutic}, or \emph{other}, and a template was developed to guide the review of papers and preprints in each category.
Following a framework often used for assessing medical literature, the review consisted of examining methods used in each relevant article, assignment (whether the study was observational or randomized), assessment, results, interpretation, and how well the study extrapolates \citep{17OQtAY4l}.
For examples of each template, please see Appendices B-D.
Another option was to contribute or edit text using GitHub's pull request system.
Each pull request was reviewed and approved by at least one other author.
Manubot also provides a functionality to create a bibliography using digital object identifiers (DOIs), website URLs, or other identifiers such as PubMed identifiers and arXiv IDs.

\hypertarget{applying-manubots-existing-capabilities-to-the-challenges-of-covid-19}{%
\subsection{Applying Manubot's Existing Capabilities to the Challenges of COVID-19}\label{applying-manubots-existing-capabilities-to-the-challenges-of-covid-19}}

Because of the ever-evolving nature of the COVID-19 crisis, many of the figures and text proposed by subject matter contributors would have quickly become outdated.
To address this concern, Manubot and GitHub's continuous integration features were used to create figures and text that could respond to changes in the COVID-19 pandemic over time.
The combination of Manubot and GitHub Actions also made it possible to dynamically update information such as statistics and visualizations in the manuscript.
When scientific writers added text that was current only as of a given date, publicly available data sources were identified whenever possible to allow the information to pulled directly into the manuscript in order to keep it up to date.
Data was pulled from a number of sources.
Data about worldwide cases and deaths from the COVID-19 Data Repository by the Center for Systems Science and Engineering at Johns Hopkins University \citep{MrwDDw9R} were read using a Python script.
Similarly, the clinical trials statistics and figure were generated based on data from the University of Oxford Evidence-Based Medicine Data Lab's COVID-19 TrialsTracker \citep{SSbnPnzT}.
In both cases, frequency data were plotted using Matplotlib \citep{1026Gxdsi} in Python.
The figure showing the geographic distribution of COVID-19 clinical trials was generated using the countries associated with the trials listed in the COVID-19 TrialsTracker, converting the country names to 3-letter ISO codes using pycountry or manual adjustment when necessary, and visualizing the geographic distribution of trial recruitment using geopandas.

GitHub Actions runs a nightly workflow to update these external data and regenerate the statistics and figures for the manuscript.
The workflow uses the GitHub API to detect and save the latest commit of the external data sources, which are both GitHub repositories.
It then downloads versioned data from that snapshot of the external repositories and runs bash and Python scripts to calculate the desired statistics and produce the summary figures.
The statistics are stored in JSON files that are accessed by Manubot to populate the values of placeholder template variables dynamically every time the manuscript is built.
For instance, the template variable \texttt{\{\{ebm\_trials\_results\}\}} in the manuscript is replaced by the actual number of clinical trials with results, 98.
The template variables also include versioned URLs to the dynamically updated figures.
The JSON files and figures are stored in the \texttt{external-resources} branch of the manuscript's GitHub repository, which acts as versioned storage.
The GitHub Actions workflow automatically adds and commits the new JSON files and figures to the \texttt{external-resources} branch every time it runs, and Manubot uses the latest version of these resources when it builds the manuscript.

\hypertarget{updating-to-manubot-in-response-to-project-demands}{%
\subsection{Updating to Manubot in Response to Project Demands}\label{updating-to-manubot-in-response-to-project-demands}}

Due to the needs of this project, project contributors also implemented new features in Manubot.
Manubot uses Zotero \citep{snhvVGmP} to extract metadata for some types of citations.
These features support directly citing clinical trial identifiers such as \texttt{clinicaltrials:NCT04292899} \citep{yTCAmOyt}.

A new plugin was also added to Manubot to support ``smart citations'' in the HTML build of manuscripts.
The plugin uses the \href{https://scite.ai/}{Scite} service to display a badge below any citation with a DOI.
The badge contains a set of icons and numbers that indicate how many times that source has been mentioned, supported, or disputed, and whether there have been any important editorial notices, such as retractions or corrections.
Using this, we were able to quickly identify references that needed to be checked again since the time they had been added.
This was invaluable given the nature of the project, where we were disseminating rapidly evolving information of great consequence from over a thousand different sources.
The badges also allow readers to roughly evaluate the reliability of cited sources at a glance.

Because in this implementation of Manubot, most collaborators were writing and editing text through the GitHub website rather than in a local text editor, we also needed to add spell-checking functionalities to Manubot.

\hypertarget{results}{%
\section{RESULTS}\label{results}}

\hypertarget{recruitment-and-manuscript-development}{%
\subsection{Recruitment and Manuscript Development}\label{recruitment-and-manuscript-development}}

Appendix A contains summaries written by the students, post-docs, and faculty of the Immunology Institute at the Mount Sinai School of Medicine \citep{cYo4O2qX, YZ4cHNuH}, and two of the authors were recruited through the American Physician Scientist Association's Virtual Summer Research Program \citep{DGTDsJZy}.

\hypertarget{data-analysis-and-integration}{%
\subsection{Data Analysis and Integration}\label{data-analysis-and-integration}}

The workflow file is available from \url{https://github.com/greenelab/covid19-review/blob/master/.github/workflows/update-external-resources.yaml} and the scripts are available from \url{https://github.com/greenelab/covid19-review/tree/external-resources}.
The Python package versions are available in \url{https://github.com/greenelab/covid19-review/blob/external-resources/environment.yml}.

\hypertarget{updates-to-manubot}{%
\subsection{Updates to Manubot}\label{updates-to-manubot}}

The scite integration and spell-checking functionalities have been integrated into the current release of Manubot .
Support for clinical trial identifiers is supported both by Manubot and by Zotero .
Using CI, Manubot now checks that the manuscript was built correctly, runs spellchecking, and cross-references the manuscripts cited in this review, as summarized in Appendix A and discussed in the project's issues and pull requests.

\hypertarget{discussion}{%
\section{DISCUSSION}\label{discussion}}

Working within the Manubot framework allowed for the successful facilitation of a massive collaborative review on an urgent topic.
Developing Manubot for the specific challenges raised by COVID-19 and expanding on both training resources resulted in seven evolving literature reviews produced by the COVID-19 Review Consortium, all of which are currently available through Manubot and, in some cases, on arXiv .
As many other efforts have described, the rate of publishing of formal manuscripts and preprints about COVID-19 has been unprecedented \citep{7ub6VM4Z}.
The Manubot framework will allow for continuous updating of the manuscripts as the pandemic enters its second year and the landscape shifts with the emergence of promising therapeutics and vaccines \citep{individual-therapeutics, i2CGFwI3}.
The manuscripts pull data from XX data sources, allowing for information and visualizations to be updated daily using CI.
This computational approach allows for the information in the manuscripts to be kept up to date automatically.

Beyond the immediate goal of applying Manubot to the challenges of COVID-19, we have also expanded Manubot to allow for broader participation in open publishing from fields where computational training in tools like version control is uncommon.
Several review articles on aspects of COVID-19 have already been published, including reviews on the disease epidemiology \citep{I2EsJmfs}, immunological response \citep{5x25saIz}, diagnostics \citep{evtsR3C5}, and pharmacological treatments \citep{5x25saIz, 18eCxyLhx} and others that provide narrative reviews of progress on some important ongoing COVID-19 research questions \citep{SAE5ME3N, xOs5ctsW}.
However, the broader topic of COVID-19 intersects with a wide range of fields, including virology, immunology, medicine, pharmacology, evolutionary biology, public health, and more, and any effort to comprehensively document and evaluate this body of literature would require insight from scientists across a number of fields.
Furthermore, during the initial phase of the COVID-19 pandemic during spring and summer 2020, and much longer in some part of the world, many biological scientists were unable to access their research spaces.
As a result, early career researchers (ECR) and students were likely to lose out on valuable time for conducting experiments.
Manubot provided a way for all contributors, including ECRs, to join a massive collaborative projects but also demonstrate their individual contributions to the larger work.

Manubot provides the advantage of allowing a manuscript to be rendered in several formats that serve different purposes, and the current project extended these options.
For example, beyond building just a PDF, Manubot also renders the manuscript in HTML and docx
The HTML manuscript format offers several advantages over a static PDF to harmonize available resources that we were able to apply to specific problems of COVID-19.
The integration of scite has made the expansive number of references more manageable by visually representing whether their results are contested or whether they have been corrected or retracted.
Cross-referencing cited preprints with their reviews in the appendix is another.
Docx is a necessary format for a biological collaboration where authors are typically not working in LaTeX.

With the worldwide scientific community uniting during 2020 to investigate SARS-CoV-2 and COVID-19 from a wide range of perspectives, findings from many disciplines are relevant on a rapid timescale to a broad scientific audience.
As a result, centralizing, summarizing, and critiquing data and literature broadly relevant to COVID-19 can help to expedite the interdisciplinary scientific process that is currently happening at an advanced pace.
The efforts of the COVID-19 Review Consortium illustrate the value of including open source tools, including those focused on open publishing, in these efforts.
By facilitating the versioning of text, such platforms also allow for documentation of the evolution of thought in an evolving area.
This application of version control holds the potential to improve scientific publishing in a range of disciplines, including those outside of traditional computational fields.

\hypertarget{acknowledgements}{%
\section*{Acknowledgements}\label{acknowledgements}}
\addcontentsline{toc}{section}{Acknowledgements}

We are grateful to Josh Nicholson and Milo Mordaunt for their support with the scite plugin, and to David Nicholson for the suggestion and feedback to enable the reporting of the locations of spelling errors in the spell-checker tool.
We thank Nick DeVito for assistance with the Evidence-Based Medicine Data Lab COVID-19 TrialsTracker data.


	\balance
	\bibliography{methods.bib}

\end{document}
\endinput
%%
%% End of file `sample-authordraft.tex'.
